\documentclass{article}
\usepackage[T1]{fontenc}
\usepackage[utf8]{inputenc} 
\usepackage{graphicx}

% Document information
\title{Technical Installation Manual}
\author{Ctrl Alt Defeat}
\date{\today} % Use a specific date or leave it empty for the current date
\begin{document}
\maketitle % Generates the title based on the information above


\section{Introduction}
The following is a technical installation manual for the project Domain Pulse, a system designed to aggregate and
process data for given domains (businesses, products, etc…),
over given sources (Google reviews, TripAdvisor and others) and provide the user,
peoples sentiment towards their domain. The system uses a Django backend, Angular frontend and PostgreSQL and MongoDB databases.
The installation of these components and the packages used within in them, shall be explained within this manual.

\subsection{Some subsection}
If we need to add more details or break it up

\section{Prerequisites}
The following are prerequisites for the complete installation and set-up of Domain Pulse: Python 3.8, Pipenv, Node.js, Angular, MongoDB, PostgreSQL.

\subsection{Python 3.8}
\subsubsection{Linux}
The following resources commnds be used for installing Python 3.8 on Linux:\\
sudo apt-get update\\
sudo add-apt-repository ppa:deadsnakes/ppa -y\\
sudo apt-get update\\
sudo apt install python3.8\\ \\
Ensure that the correct version of python is installed by running the following command:
python --version

\subsubsection{Windows}
The following resources may be used for installing Python 3.8 on Windows:
https://www.python.org/downloads/release/python-380/

\subsection{Pipenv}
Once python is installed (pip should therefore be working too) enter into the terminal:\\
pip install pipenv

\subsection{Node.js}
Node.js is needed for the installation and use of Angular and can be installed
following the instructions on the following resource:\\
https://nodejs.dev/en/learn/how-to-install-nodejs/

\subsection{Angular}
Once Node.js is installed, Angular can be installed by entering the following command into the terminal:\\
npm install -g @angular/cli

\subsection{MongoDB}
The following resources may be used for installing MongoDB
\begin{itemize}
    \item Windows: https://www.mongodb.com/docs/manual/tutorial/install-mongodb-on-windows/
    \item Ubuntu: https://www.mongodb.com/docs/manual/tutorial/install-mongodb-on-ubuntu/
    \item MacOS: https://www.mongodb.com/docs/manual/tutorial/install-mongodb-on-os-x/
\end{itemize}

\subsection{PostgreSQL 12}
The following resource can be used for installing PostgreSQL 12
\begin{itemize}
    \item https://www.postgresql.org/download/
\end{itemize}
Ensure that, regardless of what installer is used, PostgreSQL 12 specifically is installed.


\section{Installation}
Installation - this describes how the installation of your system would work

My understanding is this would be like actually getting the code onto your machine and how to install the prerequisites once you have the code
Luckily we've got stuff like pipenv install to just get all the python packages


\section{Deployment and Running}
Deployment/Running - detail how your system should be executed and provide the link to your user
manual on how the system should be used

Ideally we can write a script thta just goes and runs the frontend and backend so we don't need to write down the actual details or commands

\end{document}
