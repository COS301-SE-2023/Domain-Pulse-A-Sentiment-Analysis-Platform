\documentclass{article}
\usepackage[T1]{fontenc}
\usepackage[utf8]{inputenc} 
\usepackage{graphicx}

% Document information
\title{Technical Installation Manual}
\author{Ctrl Alt Defeat}
\date{\today} % Use a specific date or leave it empty for the current date
\begin{document}
\maketitle % Generates the title based on the information above


\section{Introduction}
Introduction - a paragraph on how the system works and what needs installation. For example, if your
system has a front and back end, you need to mention the various aspects that need to be installed
before the detailed explanation/guide

I reckon that here we give the abstract of what domain pulse is, and give an overview of how the system is structured
and what components need to be set up - will will explain the details in later sections

\subsection{Some subsection}
If we need to add more details or break it up

\section{Prerequisites}
Prerequisites - any necessary software or packages that are needed for your system to work. You
should list the requirements first. After that, specify a resource or provide instructions on how to
install the prerequisites.

Here we need to list all the software we need to install (including pipenv and npm packages). We can put links to how to install say MongoDB or Postgres.

\subsection{Some subsection}
If we need to add more details or break it up


\section{Installation}
Installation - this describes how the installation of your system would work

My understanding is this would be like actually getting the code onto your machine and how to install the prerequisites once you have the code
Luckily we've got stuff like pipenv install to just get all the python packages


\section{Deployment and Running}
Deployment/Running - detail how your system should be executed and provide the link to your user
manual on how the system should be used

Ideally we can write a script thta just goes and runs the frontend and backend so we don't need to write down the actual details or commands

\end{document}
