\documentclass[12pt]{article}
\usepackage[utf8]{inputenc}
\usepackage{graphicx}
\usepackage[a4paper,width=150mm,top=25mm,bottom=25mm]{geometry}

\title{

\\
{COS 301 User Manual}
}



\author{Ctrl Alt Defeat}

\begin{document}

\begin{titlepage}
    \centering



    \vspace{2cm}
    \hrulefill\\
    \vspace{1cm}
    {\Huge\bfseries Testing Policy Document}

    \vspace{1cm}

    {\Large Demo 4}\\
    \vspace{1cm}
    \hrulefill\\

    \vfill

    {\large Ctrl Alt Defeat}

    \vspace{1cm}


\end{titlepage}

\tableofcontents

\newpage



\section{Introduction}
\subsection{What is Domain Pulse?}
\begin{itemize}
    \item First we need to understand what Sentiment analysis is , Sentiment analysis is the process of computationally identifying and categorizing opinions expressed in a piece of text, especially in order to determine whether the writer's attitude towards a particular topic, product, etc. is positive, negative, or neutral. Now in this being said we introduce Domain Pulse.
    \item Domain pulse is the ultimate sentiment analysis platform. It gathers and analyses online opinions about any domain, be it a business, a person, or more. With stunning visuals and easy-to-understand statistics, Domain Pulse helps you understand the online presence and sentiment for any domain.
    \item Domain Pulse presents the results in a visually stunning and easy-to-understand
    format. Our wide range of visualisations bring statistics to life, which make it a breeze
    to grasp the online presence and sentiment for any domain. Take control of
    understanding public opinion like never before with Domain Pulse.
\end{itemize}
\subsection{Objectives of Domain Pulse}
\begin{itemize}
    \item The main objective of Domain Pulse is to provide a platform for users to analyse and gain valuable insight into the sentiment of any domain.
\end{itemize}
\newpage
\section{General access to the app}
\subsection{Logging into the app}
\begin{itemize}
    \item Simply add your email/username with which you registered with and your password and click on the blue 'login' button.
    \item Alternatively you can login with your Google account by clicking on the 'Sign in with Google' button.
    \item If you have forgotten your password, click on the 'Forgot Password?' link and follow the instructions.
    \item If you do not have an account, click on the 'Dont have an account?' link and register for an account using your chosen details.
\end{itemize}
\subsection{Registering for an account}
\begin{itemize}
    \item To register for an account the following information will be required:
    \begin{itemize}
        \item Username
        \item Email address
        \item Password
        \item Confirmation of password
    \end{itemize}
    \item Please enter the required information specified above in the relevant fields and click on the blue 'Register' button.
    \item If you already have an account , click on the 'Already have an account?' link and login with your registered details.
\end{itemize}
\newpage
\section{Dashboard}
\subsection{General overview}
\begin{itemize}
    \item The dashboard is the first page you will see when you log into the app.
    \item The dashboard will be the primary interactive page between the user and the app.
    \item A quick overview of the dashboard is as follows:
    \begin{itemize}
        \item On the left hand side of the page you will see a blue side bar 
        \item On the top of the page you will see the list of sources for chosen domain
        \item On the top right of the page are the actions that can be performed on certain components on the dashboard or the entire dashboard
        \item The center console of the page is where the insights and statistics will be displayed 
        \item On the bottom left you will see all the relevant graphs and charts displaying the statistics of the chosen source 
        \item On the bottom right of the page is sample data
    \end{itemize}
    \item A more detailed explanation of the dashboard will be given in the following sections
\end{itemize}



\end{document}
